\chapter{О предмете этой книги}
\section{Классификация платформ по Таненбауму}

В своей книге "Архитектура компьютера" Таненбаум предложил иерархическое описание компьютерных платформ.\footnote{Эндрю Таненбаум (Andrew Tanenbaum) -- профессор  Амстердамского Свободного Университета, известный как "дедушка Линукса", а также как автор ряда учебников по организации вычислительных систем,  в том числе книги "Structured Computer Organisation" (в русском переводе -- "Архитектура компьютера").  В этой книге мы представляем несколько упрощенную версию предложенной Таненбаумом классификации, но надеемся, что она соответствует намерениям оригинала.}.  
В соответствии с этим описанием, ресурсы, предоставляемые платформой могут использоваться для решения определенного множества проблем или для поддержки других (более высокоуровневых) платформ.  Мир вычислений по сути своей иерархичен.  Это в равной степени относится и к анализу проблем, и к дизайну машин для решения этих проблем.  Вам полезно было бы ценить иерархии везде, где вы их находите.  Они полезны как познавательный инструмент, но, кроме того, они позволяют вам скрывать сложность под покрывалом абстракции.  По мере того, как мы будем двигаться вверх по иерархии платформ, мы будем иметь дело с предметами, все менее и менее зависимыми от определенной аппаратной реализации, и более доступными пониманию человека.

Предложенная Таненбаумом классификация платформ имеет нумерованные уровни, у которы больший номер означает, что платформа предлагает более высокий уровень абстракции и пользуется средствами, которые предоставляются платформами более низкого уровня.  Внизу (Уровень 0) мы находим аппаратные устройства, каждое из которых решает настолько крошечную задачу, что для достижения чего-то значимого нам необходимы тысячи или даже миллионы таких устройств.  Но без этих устройств ничто не будет работать.  Вверху (на Уровне 5 и выше) мы находим виртуальные машины выполняющие определенные прикладные задачи.  Удобно визуализировать иерархию Таненбаума в виде диаграммы и разместить представляющие интерес платформы на одном рисунке. (см. рис. \ref{levels}).
\begin{figure}
\caption{Иерархия платформ}\label{levels}
\end{figure}

Замечание: В последние десятилетия все чаще приходится сталкиваться с ситуацией, когда некоторые платформы высоких уровней заняты не решением прикладных задач или предоставлением высокоуровневых абстракций, а имитацией поведения платформ более низкого уровня.  Так, платформа Уровня 0 CdM-8 -- это не транзисторы и провода из кремния и меди, а виртуальные вентили и соединения, предоставляемые платформой уровня 5  -- программным эмулятором {\tt logisim}.  

Еще один пример -- это виртуальные машины в общепринятом современном смысле, например, те, которые создаются платформой VirtualBox или qemu.  Платформа VirtualBox (так называемый \emph{гипервизор}) -- это программный комплекс Уровня 5, который, используя ряд остроумных трюков и специализированных сервисов платформы Уровня 2, имитирует работу платформы Уровня 1\nicefrac{1}{2} (PC-совместимый компьютер без операционной системы).  Это позволяет одновременно запустить несколько разных платформ Уровня 2 (\emph{гостевых операционных систем}) на одной машине.

Причины, по которым разработчики платформ используют эти нарушения иерархии, отличаются большим разнообразием.  Например, это могут быть соображения удобства отладки и тестирования, или желание обеспечить совместимость со старыми платформами, или желание запустить несколько платформ на одной машине.

Программная эмуляция и виртуализация низкоуровневых платформ -- это очень интересная тема, но она увела бы нас далеко в сторону от темы книги. 

\section{Каждая платформа имеет язык}

Платформы отличаются не только наборами ресурсов, которые они предоставляют, но также и средствами, используемыми для описания задач, которые мы хотели бы решить при помощи платформы.  Иными словами, каждая платформа имеет собственный \emph{язык}.

Понятие языка вычислительных платформ несколько отличается от обыденного значения слова "язык".  В вычислительной технике, (формальный) язык определяется как набор правил сочетания символов.  Сочетания, соответствующие этим правилам, формируют валидные предложения.  Вместе, набор символов (\emph{алфавит}) и правил (\emph{синтаксис}) определяют язык и называются его \emph{грамматикой}.   Последовательность символов, которая не подчиняется правилам языка,  \emph{не является предложением} и называется \emph{недопустимой} или \emph{синтаксически некорректной}.

\subsection{Программируемая платформа имеет язык программирования}

Нам гврвнтируется, что платформа "поймет" все валидные предложения, написанные на ее языке.  Если платформа \emph{программируемая} (как Платформа 3\nicefrac{1}{2} CdM-8), ее язык называется \emph{языком программирования}\footnote{Платформа может иметь управляющий язык, который не является языком программирования.  Точное определение того, какие синтаксические и семантические средства необходимы, чтобы язык считался языком программирования, выходит за пределы темы этой книги. Следует также отметить, что граница между языками программирования и "просто" языками управления не всегда строго определена. Так, язык \LaTeX, на котором написана эта книга, обычно используют и рассматривают как "просто" язык разметки текста.  Но он имеет достаточно мощные средства описания последовательностей действий, чтобы на нем можно было написать бесконечный цикл или саморазмножающуюся программу (\emph{вирус}).}, а валидные предложения называются \emph{программами}.  Последовательность символов, которая не формирует валидного предложения  \emph{не является программой} и не может быть "понята" платформой.  Однако, это только часть истории, потому что грамматические правила не говорят нам, \emph{что платформа будет делать} при попытке исполнить программу (что эта программа \emph{означает}).

Существует еще один набор правил и описаний, предназначенный для того, чтобы присвоить смысл программам.  Смысл программы называется ее  \emph{семантикой}.  Программа может быть синтаксически корректной (и этого достаточно, чтобы она могла быть исполнена), но некорректной семантически (делать не то, что мы хотели).  В этом случае говорят, что программа содержит \emph{семантические ошибки} или "баги" (в русском языке, слово "ошибка" без уточнений можетозначать как синтаксическую, так и семантическую некорректность).

Замечание: Любая работающая программа, предоставляющая ресурсы, которые могут быть использованы для решения проблемы, сама по себе является платформой.  Однако, вовсе не обязательно такая платформа является программируемой.  Например, простой текстовый редактор (такой, как Windows notepad) -- это платформа для создания и редактирования текстовых файлов, теоретически пригодная и для написания текстов на языках программирования.  Но эта платформа не имеет  собственного языка программирования.  Более сложные редакторы, такие, как  vim или Notepad++ -- это программируемые платформы.  Они имеют \emph{скриптовые языки} или \emph{языки сценариев}, при помощи которых можно механизировать сложные или повторяющиеся действия, например, реализовать синтаксическую подсветку или "красивое" форматирование программ на языке С.  В соответствии с классификацией Таненбаума, сам текстовый редактор является платформой Уровня 5.  Исполняющиеся в нем скрипты, очевидно, следует считать платформой Уровня 6 или выше.

\subsection{"Язык" электронных схем?}

? ?????????? ???????????, ?? ?????????? ??????????? ????? ??? ???????? ?????, ??????? ????? ?????? ??????.  ???? ??? ? ???????? ????????, ?? ????? ???? ??????? ??????????? ????, ??? ?????? ????????? -- ??? ?????? "????? ????", ? ??????????? ?? ??, ??? ??? ?????????? ????? ??????????? -- ??? "?????????????? ???????".  ?????? ??? ??????? ????? ?????? ???? ??????????.  ?????? ??????????? ?????????? ?????? ???????? ??? ?????????, ??????????????? ????????? ?????????? ? ?? ??????????.  ????????? ?????? ?????????? ????? ???????? ??????????? ?????, ??????? ????????????? ?????????.  ?????, ??????? ??????????? ???????? (?, ????? ???????, "????????????? ?????????") ????? "????????" ? ??? ??????, ??? ??? ???????? ?????-?? ???????? ??? ???? ?????????????????? ????????.  ?? ??????? ?????????? ??????????? ???? ?? ???? ?? ??????? ???, ??? ????? ????? ????? ??????.

????????? ??????? ?????????? ??? ????????? ????????? ???????? ? ???????, ??????? ????????? ?? ???????????? ?????????, ????? ???? ??????? ??? "?????????" ?????.  ??? ??, ??? ? ?????????, ????? ?????????????? ? ????????? ?????, ??????? "????????????? ?????????" (????????? ????????? ????? ????????), ?? ??????????? ???????????? (??? ????????? \emph{????????} ????? ????????).  ????? ?????? ???? ????????? ?????????? \emph{??????}, ????? ??? ??, ??? ? ???????????? ???????????? ?????????.

\section{??? ????????? "?????????"}

???????? ?? ????????? ????????, ?????????? ???? ????? ???? ????? ????? ??????????, ?????????? ? ???????????? ?????????: ??? ??? ?????????? ?????????????? ????? ????????????? \emph{?????? ??????????}.  ??????? ?????? ???????? ????????? ?? ??????? ????????? ??????????: ?????, ???? ??????, ??????? ???????????.  ?????? ?????? ???????? ?????????? ???? ????????: ????, ???????, ???????, ????????.

???????? ????????? ????????????? ?? ??????: ????? ????????? ?????? 2 ????? ????? ??????? ?????? ???????????, ???????? ????????? ????????? ? ?????????? ? ???????????? ????? ??????, ?? ????????? ?????? 5 ????? ????????????? ?????????? ??????? ??????????? ???-???????? ??? ?????? ??? ????????? ?????? ? ???? ??????.  ??????????????? ?????????? ?????? ???? ??????????? ?? ????????? ?????????? ????? ??????? ??????.  ????? ???????? ???????? ????? ????????? ????? ??????????, ????, ? ????? ??????, ????????? ??????? ?????? ?? ??????? ?????????????????? ?? ????????? ????? ??????? ?????????? ??? ??????????.

\subsection{ ??????????, ????????? ? ??????????? ??????}

??????????? ? ????????? ??????????, ?? ?????, ??? ???????????? ????? ???????????? ????????? ??? ???????? ????? ???????? ??????, ??????????????? ??????????? ????? ?????? ???????.  ?????????????? ????????? ????????? ??? ????????? ????? ??????????????? ?????????, ??????? ????????????? ??????????? ????????, ???????, ? ???? ???????, ??????????? ????? ?????? ??????? ? ???? ????????? ??????????, ???????????? ????????? ?????? ????????.  ??????????????? ????????? ????????? ????? \emph{????????????}, ?????? ??? ?? ?????? ???????? ????????????, ?? ? ???????????? ???? ???????? ?? ??????? [??? ?????] ?????? ? ??????? ????, ????? ????????? ??????? ?????? ??????????? ????????? ????????.

????? ????????? ????????? ?????????? ??????????, ?? ???????, ??? ??? \emph{??????????? ??????} (??), ? ?? ???????? ??????? ???????? ????? ????????? \emph{??????????????}.  ?? ????? ? ????, ??? ??????????????? ????????? ??????? ??????? "????????" ??????, ??????? ????????????? ????? ??????? (?? ?????? ????? ??????? ? ?????????????) ???????? ? ???????, ??? ?????????????? ?????????, ?? ??????? ??? ????????.  ????? "???????????" ?? ?????????? ??? ????, ????? ???????????, ??? ??? ??????? ??????????? ??????????, ? ?? ?????????.

???????????? ???????? ????????? ????? ????????? ?????? ?????? ??????????????? ?? ?????????? ???????????, ??? ??????-???? ??????? ???????????? ???????????.  ????????, ?? ????? ?? ????????? ?????? ??? ????????? ??????? (????????? ?????????) ?? ?????????? ????????.  ??? ??? ?? "????????" ????????? ?????????.  ??????, ????? ?????? ????? ?? ??????????? ?????????? ?????????, ??? ??????? ?? ?? ?????????????? (?, ?????? ?????, ????????????) ????? ???????.  ? ??? ????? ?? ??????????? ?????????? ???? ?? ???????? ?????????? ??????????, ??? ???????? ? ????? ????? ??????.  ? ??? ???? ?? ???????? ?????? ??? ????? ??????, ??? ??? ???, ????????, ???????? ?? ??????? ?????? ?????? ??? ???????? ? ????????? ??????????? ?????.

??????? ????? ????????? ??????? ????????????? ?????????? ????????? ? ???????? ??? ??? ?????????, ??????? ?????????? ?? ????? ???? ???, ??? ????? ??? ????????? ?????????.  ?? ???????? "???????????" ????????? ?????????, ??????? ?????? ????? ?? ?? ?????, ??? ? "????????", ?? ?? ????????? ???? ???? ????? "?????????" ??????????.  ???? ?????????? ????????? ????????????? ????????? ????? ???????? ? ???? ?? ???? ???????????? ?? ??????????, ??????? ?? ????????????, ?? ????? ???????? ????????????, ?? ?????????? ?????? ????? ??????????? ??????.  ?? ????? ????? ???????? ?????????? ????? ??, ?? ??????????? ? ?????????????? ???????? ??? ???????? ????????????.  ????? ????, ?? ????? ???????? ????? ?????? ???????? ??? ??? ?? ????? ?????????? ?????????, ?????? ?? ??????? ????????? ?????? ??????????????? ????????? ? ???? ??????????? ??????, ??? ??? ?? ????? ????? ?? ?????? "???????????" ????????? ?????????, ?? ? "???????????" ???-???????, "???????????" ????? ??? ??????????? ? ?.?.

\section{?????????? ????? ???????? ??????????}

?????????? ??? ???????? ????????? ??? ???????? ??????? ????????, ??????????????? ???????????????? ???????????, ? ????? ??????? ????????, ??????????????? ??????? ???????? ????????.

\subsection{?????????}

????????? -- ??? ???????????? ????????, ??????? ????? ???????????? ??? ?????????? ?????????? ????????????. ????????? -- ??? ?????????? ??????????? ?????? ??? ???????? ??????????? (????????) ????, ???????, ? ???? ???????, ????? ?????????????? ??? ???????? ??????? ??? ????????? ????-?? ??? ????? ????????.  ??? ???????????? ???????? ? ????????? ? ?????????????.  ????????, ??? ???????????? ????? ?????????? ????, ??????, ????, ????? ? ?.?.  ?????? ?? ???? ??????????? ??? ?? ???? ???????????? ???????? ????? ??????? ???????????, ?????, ??? ??????? ? ?????.  ?? ??? ???????? ? ???? ????, ??? ????? ?????? ?? ?????????? ????? ??? ? ???-?? ?????????. ????????, ?? ????? ???????????? ????????? AA ??? ????????? ????????? ????????, ? ????? ?????? ? ??????????? ??? ???????????, ??????? ???????????? ??? ???????????? ???? ?????????.  ??????, ???? ?? ????? ?????????? ?????????, ??? ?????????? ????? ?????????? ?????????? ? ????????? ?? ?????????? ???????, ????? ???????? ?????????\footnote{??? ??????????.  ???? ??????? ??????????? ????????????? ??? ?? ????????????? ? ????????? ?? ?? ???????, ????? ????????? ????? ?????????????? ?????, ??? ????????? ????????? AA}.

? ???????????? ????????????, ??????????? ???????????? ? ??????????, ????? ????????????? ? ???????? ?????, ? ????? ????? ???? ???????????? ? ?????????, ??????? ????? ????????? ??????????? ???????????.  ????? ??? ????? ??????????, ??????????? ???????????? ????? ???????, ?? ?????? ???????? ?????????? ? ?????, ? ??? ?? ????? [?????????] ???????????? ? ???, ??? ??? ?????????? ???????????????.

????????? ????? ???????????? ??? ???????? ??? ?????????????? ??????????? ???????????.     ? ????????????????, ????? ????????? -- ??? ???????? ?? ????????? ??????????.  %?????????? ??????????? ????? ??????????? ?????????, ????????? ??? ?????. 
??? ???????? ??????? ????????? ??????????? ?????????? -- ??? ????? ??????????? ? ??????????? ?????? ?????????, ?????????? ?????????????????.  ???????????? ? ??? ??? ???? ????? ?????????? ?? ???? ??????????? ?????? ????????????????, ??????? ???, ??? ?????? ????????????????, ? ???? ????? ??? ??? ????? ?????? ???? ???????.  

?????? ???????????????? ?????? ????? ???? ?????? ?? ?????? ??????????? (? ????? C ??? ???? ??????????), ?? ??? ???? ?? ?????????? ???????? "??????".  ?????????? ?? ???????????????? ?????? ????????????? ?? ?????, ??? ? ???? ????????? ??? ?????? ?????.  ???? ?? ? ????? ????????? ??????? ???????????????? 5 ???, ?? ??? ??????? ????? ????????????? 5 ???.  ???????? ?? ???, ? ???????????????? ?? ????? ?????????? ???????????????? ?????? ????????????.  ????????? ?? ?????? ????? ?? ?????? ? ???? ???????? ????????? 3\nicefrac{1}{2}.   

????? ????????, ??? ????????? \emph{???????????}.  ??? ??????? ? ?????????? ?????? ?????????? ?????? ???, ????? ????? ????????? ?????????.   ???? ????? -- ??? ???????? ?? 1000 ????????, ??? ???? 1000 ???????? ??? ?????????? ????? ????? ? ?????? 1000 ????????  -- ??? ?????????? ?????? ????? ??.  ?? ????????? ????????? ???? ????? ? ????????? ??? ???????????? ?? ?????? ????.

??????????, ???? ?? ????? 16Gb ??? ? ??????????, ??? ???????? ?????????? ????? ?????? ????????? ??????, ??????? ? ???????????? ????? ????????? ????? ?????? ????????????, ?????????? ????? ?????? ???????? ? ?????? ????? ??????, ??? ???? ?? ????? ???????? 8Gb ???.

\subsection{?????????????}

????????????? -- ??? ?????? ???????????? ???????? ?????????? ??? ?? ?????????? ????? ???????????? ? ??????????? ????????????, ??? ? ????? ???????? ???????????? ???????????.   ????????????? ??????? ? ???, ??? ?????? ????????? ??????? ????????? ???????? ?????? ????????????? ?????????? ??????? ????, ??????? ????? ???????? ?????????? ????? ????? ??????, ????????? ??, ? ????????? ? ??????? ? ?????????? ????????? ??????????.  

?????????? ?????? ?????? ??? ???? ?????????? ?????????? ??????????? ?????????????? ????????? ???????.  ??????????, ??? ?????? ? ???????????? ? ???????????? ? ???????? ?????????, ??????? ?????????? \emph{???????????????}.  ???, ?????????? ????????? ?????????????? ????????? ?? ???????? ?????, ? ???-??????? ?????????????? HTML-????????, ??????? ???????? ?? ??????? ?????? ?? ??????? ? ??????????? ? ??????? ????????????? ???-???????? ?? ???? ?????????.  

????????????? ? ?????? ???? -- ????? ?? ???????????? ????????, ??? ? ????????????????: ?????? ???, ????? ????????????? ????????? ???? ? ?? ?? ???????, ?? ????????? ???? ? ?? ?? ?????????????????? ???????? ?? ?? ??????? ?, ??????????, ??????????.  ???? ????????? ?????? 2 CdM-8 -- ?????? ????? ??????? ?????????????.  ??? ?????????? ????? ? ??? ?? ???????, ??? ?????? ?????????? ???? ? ??? ?? ????? ???????? ? ????? ? ??? ?? ??????????????????.

?????????? ????? ??????? ??????????????, ??????? ?????????? ????????? ?????????? ????? ??? ????? ???????????? ????????????? ????????, ??????? ???????, ??? ???????? ???????.  ??? "???????????" ?? ????????? ?????????? ??????, ???????? ???????????? ?????????? ??? ????????? ????????? ?????????? ????????????, ????????? ????????????? ?????????? ? ??????????? ??????? ?????? (\emph{????}) ? ??..  ?????? ??? ???????? ??????????? ??????????? ??????????? ???????? ?????? 2 -- x86, ARM.  ?????????? ???? ?????????? ?? ???? ?????? ? ????? ????????? ??????? ?? ? ?? ????????????? ??????????, ??? ??????? ?? ?? ??????????? ??? ????????.
